\chapter{Related Work}

\section{Verwandte Arbeiten}
Mit weiteren Anwendungsfällen beschäftigt sich das ,,Fraunhofer Institut BIG DATA'' in ihrem Paper [\href{https://www.bigdata-ai.fraunhofer.de/content/dam/bigdata/de/documents/Publikationen/KI-Potenzialanalyse_2017.pdf}{Big Data Frauenhofer}].\vspace{0.2cm}

Um den Einsatz künstlicher Intelligenz im werte-orientierten Marketing zu bewerten, befassten sich in [\href{https://www.econstor.eu/bitstream/10419/222610/1/1725938928.pdf}{EConster}] einige Professoren des ,,Leibniz-Informationszentrum Wirtschaft'' mit diesem Thema.\vspace{0.2cm}

Die Masterarbeit [\href{https://reposit.haw-hamburg.de/bitstream/20.500.12738/7932/1/master_thesis.pdf}{Bitstream}] von Eduard Weigandt befasst sich mit der Personalisierung im E-Commerce basierend auf Data-Mining. Interessante Grundlagen zum Einstieg in E-Commerce und künstlicher Intelligenz sind auf [\href{https://www.epoq.de/blog}{Eqop}] zu finden. Die Firma Kobold AI befasst sich in ihrem Artikel [\href{https://www.kobold.ai/kundensegmentierung-ki}{Kobold.ai}] ,,Optimale Segmentierung von Bestandskunden durch KI'' und erläutert Methoden zur Clustering von Bestandskunden. ,,Datasolut'' ist ein weiteres Unternehmen das sich in [\href{https://datasolut.com/kundenklassifizierung-definition-vorteile-und-methoden}{Datasolut1}] mit Kundenklassifizierung, Clusteranalyse und maschinellem Lernen befasst. Ebenfalls von ,,Datasolut'' ist der Artikel [\href{https://datasolut.com/ki-im-e-commerce}{Datasolut2}] in dem erfolgreiche Anwendungen und Beispiele zum Thema künstlicher Intelligenz im E-Commerce aufgezeigt werden.

