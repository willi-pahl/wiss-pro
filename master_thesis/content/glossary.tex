\newglossaryentry{stakeholder}
{
name=Stakeholder,
description={Stakeholder oder Projektbeteiligte sind alle Personen, Institutionen und Dokumente, die von der Entwicklung und vom Betrieb eines Systems in irgendeiner Weise betroffen sind. Dazu gehören auch Personen, die nicht in der Systementwicklung mitwirken, aber das neue System zum Beispiel nutzen, in Betrieb halten oder schulen}
}
\newglossaryentry{verantwortungsdiffusion}
{
	name=Verantwortungsdiffusion,
	description={bezeichnet das Phänomen, dass eine Aufgabe, die offensichtlich zu tun ist, nicht ausgeführt wird – obwohl dafür genügend fähige Personen anwesend beziehungsweise verfügbar wären}
}
\newglossaryentry{planning_poker}
{
	name=Planning Poker,
	description={als eine Variante der Aufwandsschätzung ist in den Scrum-Kreis\-lauf eingebettet. Dabei kommt es in den Sprint Planning Meetings zur Anwendung, meist wenn ein neues Product Backlog geschrieben worden ist. Beim Planning Poker ist man darauf bedacht, spielerisch die Erreichung von Konsens innerhalb eines Teams zu erwirken}
}
\newglossaryentry{storyboards}
{
	name=Storyboards,
	description={ähneln Szenarien mit denen Interaktion veranschaulichen werden, die erforderlich sind, um ein Ziel zu erreichen. Anstatt eine Liste von Schritten dazustellen, visualisiert ein Storyboard die Interaktion ähnlich wie ein Comic}
}
\newglossaryentry{personas}
{
name=Personas,
description={stellt einen Prototyp für eine Gruppe von Nutzern dar, mit konkret ausgeprägten Eigenschaften und einem konkreten Nutzungsverhalten}
}
\newglossaryentry{personas_driven_user_stories}
{
	name=Personas-driven User Stories,
	description={sind User Stories die eine konkrete Persona verwenden und die User Storie wird für eine bestimmte Personengruppe, die im Persona definiert wurde, erstellt. Es wird also nicht \glqq Als Anwender möchte ich ...'', verwendet, sondern beispielsweise \glqq Als Peter möchte ich ...''}
}
\newglossaryentry{low_fi_prototypen}
{
	name=Low-Fi Prototypen,
	description={ist ein Prototyp bei dem der Fokus auf Benutzerführung und Funktionalität weitestgehend losgelöst vom Design liegt.  Es ist eine (interaktive) Präsentation eines digitalen Produkts, wie beispielsweise einer Website oder einer App, die möglichst weit am Anfang des Entwicklungsprozesses steht, mit nur einigen viuellen Aspekten des Endproduktes}
}
\newglossaryentry{countinuous_integration}
{
	name=Countinuous Integration,
	description={ist ein Begriff aus der Software-Entwicklung, der den Prozess des fortlaufenden Zusammenfügens von Komponenten zu einer Anwendung beschreibt. Das Ziel der kontinuierlichen Integration ist die Steigerung der Softwarequalität}
}
\newglossaryentry{continuous_delivery}
{
	name=Continuous Delivery,
	description={bezeichnet eine Sammlung von Techniken, Prozessen und Werkzeugen, die den Software-Auslieferungsprozess verbessern. Die Automatisierung der Integrations- und Auslieferungsprozesse ermöglicht schnelle, zuverlässige und wiederholbare Deployments. Erweiterungen oder Fehlerkorrekturen können somit mit geringem Risiko und niedrigem manuellem Aufwand in die Produktivumgebung oder zum Kunden ausgeliefert werden}
}
\newglossaryentry{ux_schulden}
{
	name=UX-Schulden,
	description={bezeichnet die Kluft zwischen aktueller UX und der künftigen UX, die nach Begleichen der UX-Schulden vorhanden sein könnten}
}
\newglossaryentry{technischen_schulden}
{
	name=technischen Schulden,
	description={Technische Schulden oder Technische Schuld (englisch technical debt) ist eine in der Informatik gebräuchliche Metapher für die möglichen Konsequenzen schlechter technischer Umsetzung von Software. Unter der technischen Schuld versteht man den zusätzlichen Aufwand, den man für Änderungen und Erweiterungen an schlecht geschriebener Software im Vergleich zu gut geschriebener Software einplanen muss}
}
\newglossaryentry{hci}
{
	name=HCI,
	description={(\textit{englisch Human–computer interaction}) erforscht das Design und die Verwendung von Computer-Technologie an der Schnittstelle zwischen Menschen (Anwendern) und Computern. Forscher auf dem Gebiet der HCI beschäftigen sich mit der Art und Weise, wie Menschen mit Computern und Design-Technologien interagieren}
}
\newglossaryentry{heuristische_evaluation}
{
	name=Heuristische Evaluation,
	description={(Heuristik – zu griech. heuriskein ‚finden‘) ist eine Methode, um die Gebrauchstauglichkeit einer Benutzeroberfläche vor Fertigstellung des Gesamtsystems zu beurteilen}
}
\newglossaryentry{content_management_system}
{
	name=CMS,
	description={Nutzerfreudliche Bedienungsoberfläche einer Software}
}

\newglossaryentry{relaunch}
{
	name=Relaunch,
	description={Bei einem Relaunch handelt es sich um eine Wiedereinführung bzw. den Neustart beispielsweise eines Produkts, einer Website oder einer Marke. Der Begriff setzt sich aus den Bestandteilen ,,re'' (wieder/neu) und ,,launch'' (Start/Einführung) zusammen}
}
\newglossaryentry{corona}{
	name=Corona,
	description={Oder auch COVID-19 (Coronavirus disease 2019) ist eine Infektionskrankheit, die durch SARS-CoV-2 verursacht wird. Die Übertragung des Virus findet vor allem über die Luft (Tröpfcheninfektion), ferner über Hände und Gegenstände (Schmierinfektion) statt. Die Krankheit kann mit Fieber, Husten, Atemnot, Geschmacksverlust, Durchfall und Müdigkeit einhergehen. Zuweilen entwickelt sich eine Lungenentzündung. Die Krankheit breitete sich ab Dezember 2019 von der chinesischen Millionenstadt Wuhan her aus, vermutlich nachdem das Virus auf einem dortigen Tiermarkt von einem Tier auf einen Menschen übergesprungen war}
}
\newglossaryentry{eda}{
	name=EDA,
	description={}
}
\newglossaryentry{storytelling}{
	name=Storytelling,
	description={}
}
\newglossaryentry{unique_selling_point}{
	name=Unique Selling Point,
	description={}
}
%\newglossaryentry{}{name=CMS,description={}}
