\chapter{Einführung}
Das Volumen der bestellten Waren im E-Commerce Bereich wächst stetig. Die Corona-Krise hat diesen Prozess noch beschleunigt. Die meisten Umsätze verzeichnen virtuelle Marktplätze, wie Amazon und Ebay. Aber auch kleine Onlineshops können durchaus bestehen.
Im E-Commerce nutzen Onlineshop Betreiber Kundenklassifizierungen, um Kunden einzuteilen und ihnen besser Angebote unterbreiten zu können. Oft sind diese Klassifizierungen in den ersten Schritten bei der Shop-Planung oder bei einem Relaunch erstellt wurden, basierend auf einer erwarteten und anvisierten Kundengruppe. In vielen Fällen werden keine Kundenklassifizierungen erstellt. Es wird unterstellt das alle Menschen zur Zielgruppe gehören.
Mithilfe von Machine Learning können aus vorhandenen Daten, Nutzergruppe bestimmt werden.

\section{Motivation}
Hier kommt auch die Problembeschreibung.
\section{Ziele der Arbeit}
\section{Inhaltlicher Aufbau der Arbeit}
