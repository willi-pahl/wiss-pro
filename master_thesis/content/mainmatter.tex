\chapter{Einführung}
\section{Motivation}
Hier kommt auch die Problembeschreibung.
\section{Ziele der Arbeit}
\section{Inhaltlicher Aufbau der Arbeit}

\chapter{Grundlagen}
\section{Grundbegriffe}
\textit{Big Data}\vspace{0.1cm}

Als Big Data werden Daten bezeichnet, die entweder zu groß, zu komplex, zu schnelllebig oder zu schwach strukturiert sind, um diese mit herkömmlichen Methoden auszuwerten. Big bezieht sich in der Definition auf die vier Dimensionen. Auf volume (Umfang, Datenvolumen), velocity (Geschwindigkeit, mit der die Datenmengen generiert und transferiert werden), variety (Bandbreite der Datentypen und -quellen) und veracity (Echtheit von Daten).\vspace{0.5cm}

\textit{Knowledge Discovery in Databases}\vspace{0.1cm}

Dies hat das Ziel aus vorhandenen meist großen Datenbeständen, fachliche Zusammenhänge zu erkennen. Zu den Teilschritten des KDD Prozesses gehören 1. Bereitstellung von Hintergrundwissen, 2. Definition der Ziele, 3. Datenauswahl, 4. Datenbereinigung, 5. Datenreduktion, 6 Auswahl eines Modells, 7. Data-Mining, die eigentliche Datenanalyse, 8. Interpretation der gewonnenen Erkenntnisse.\vspace{0.5cm}

\textit{Data-Mining}\vspace{0.1cm}

Data-Mining die systematische Anwendung von statischen Methoden auf große Datenbestände, um neue Querverbindungen zu erkennen.

\section{Verwandte Arbeiten}
Es gibt sehr viele Arbeiten.
\section{Big Data}
Vorbereiten der Daten, sammeln, Auswerten und Bereinigen, sowie Zusammenführen.
\section{Data-Mining}
\section{Clustering}

\chapter{Kern der Arbeit}
\section{Probleme und Lösungsansätze}
\section{Methodiken und Vorgehen}
\section{Architektur}
\section{Algorithmen}
Hier wird der k-Means-Algorithmus erläuter.

\chapter{Implementierung}
\section{Umsetzung der Datenverarbeitung}
So was wie Daten bereinigen und zusammenführen.
\section{Umsetzung der Clusterung}

\chapter{Evaluation}
\section{Ausbau der Umgebung}
Hier könnte das \Gls{content_management_system} erwähnt werden.
\section{Ergebnisse}
\section{Bewertung und Diskussion}

\chapter{Zusammenfassung}

\chapter{Ausblick}
